% L�gg till: An Effective Design of Master-Slave Operating System Architecture for Multiprocessor Embedded Systems
\section{Introduction}

With the advent of multicore and simultaneous multithreaded processors, shared
memory multiprocessors are becoming very common. This makes operating system
support for multiprocessors increasingly important. Some operating systems
already include support for multiprocessors, but many special-purpose
operating systems still need to be ported to benefit from multiprocessor
computers. To get good multiprocessor support, a number of challenges
involving concurrency issues, have to be solved, and how these are handled
affect the engineering time needed for the multiprocessor implementation.

There are a number of possible technical approaches when porting an operating
system to a multiprocessor, e.g., introducing coarse or fine grained locking
of shared resources in the kernel, introducing a virtualization layer between
the operating system and the hardware, or using the master-slave and other
asymmetric approaches.

The development effort and lead time for porting an operating system to a
multiprocessor vary depending on the technical approach used. The technical
approach also affects multiprocessor performance.  The choice of technical
approach depends on a number of factors; two important factors are the
available resources for doing the port and the performance requirements on the
multiprocessor operating system. Consequently, understanding the performance
and development cost implications of different technical solutions is crucial
when selecting a suitable approach.

In this paper we identify seven technical approaches for doing a
multiprocessor port. We also provide an overview of the development time and
multiprocessor performance implications of each of these approaches.  We base
our results on a substantial literature survey, and also provide a case study
concerning the development effort and multiprocessor performance (in terms of
scalability) of different versions of Linux. We have limited the study to
operating systems for shared memory multiprocessors.

The rest of the paper is structured as follows.
Section~\ref{sec:survey:challenges} describes the challenges faced in a
multiprocessor port. In Section~\ref{sec:survey:categorization} we present a
categorization of porting methods, and
Section~\ref{sec:survey:implementations} then categorizes a number of existing
multiprocessor systems. In Section~\ref{sec:survey:linux}, we describe our
Linux case study, and finally discuss our findings and conclude in
Section~\ref{sec:survey:discussion}.

\section{Multiprocessor Port Challenges}
\label{sec:survey:challenges}
A uniprocessor operating system needs to be changed in a number of ways to
support multiprocessor hardware. Some data structures must be made CPU-local,
like for example the currently executing process. The way CPU-local data
structures are implemented varies. One approach is to replace the affected
variables with vectors that are indexed with the CPU number. Another is to
cluster the CPU-local data structures in a virtual memory region and map this
region to different physical memory pages for each CPU.

In a uniprocessor kernel where processes cannot be preempted in-kernel, the
only source of concurrency issues is interrupts, and disabling interrupts is
then enough for protection against concurrent access. On multiprocessors,
disabling interrupts is not enough as it only affects the local processor.
Also, multiple processors accessing shared memory at the same time can cause
conflicts even outside interrupt context. A locking scheme is therefore
needed. Locking can be implemented in different ways, e.g., through spinlocks
or semaphores. Spinlocks are implemented by letting the processor do a busy
wait for a held lock. Acquiring a semaphore usually involves a context switch
to let other processes execute, and is therefore better suited if the lock is
held for a long time. Kernels which support in-kernel process preemption
(i.e., involuntarily suspending threads within the kernel) need protection of
shared data structures even on uniprocessors.

Locking affect both performance and development time of multiprocessor
operating systems. A held spinlock means that another processor cannot enter a
section of code or access a region of data, which locks the processor out from
useful work. At the same time, race conditions can occur if data is not
properly protected by locks, and careful examination of dependencies between
locks etc. adds to the development time.

Many modern processors employ memory access reordering to improve performance.
Therefore, memory barriers are sometimes needed to prevent inconsistencies.
For example, a structure needs to be written into memory before the structure
is inserted into a list for all processors to see the updated data
structure. Further, debugging, which is hard even in uniprocessor operating
systems, become more difficult when several processors execute in the kernel
at the same time.


\begin{figure*}[t!]
  \begin{tabular}{ccc}
    \epsfig{width=0.30\linewidth, file=figures/smp-survey/giant_lock}\hspace{0.22cm} & \epsfig{width=0.30\linewidth, file=figures/smp-survey/coarse_grained}\hspace{0.22cm} & \epsfig{width=0.30\linewidth, file=figures/smp-survey/fine_grained} \\
    (a) Giant lock & (b) Coarse-grained & (c) Fine-grained \\
    \\
    \epsfig{width=0.30\linewidth, file=figures/smp-survey/master_slave}\hspace{0.22cm} & \epsfig{width=0.30\linewidth, file=figures/smp-survey/appkern}\hspace{0.22cm} & \epsfig{width=0.30\linewidth, file=figures/smp-survey/piglet} \\
    (d) Master-slave & (e) The Application kernel & (f) Piglet \\
  \end{tabular}
  \caption[Multiprocessor operating system oganizations.]{\emph{Continued on
      next page}}
  \label{fig:survey:methods}
\end{figure*}

\begin{figure*}[t]
  \begin{tabular}{ccc}
    \epsfig{width=0.30\linewidth, file=figures/smp-survey/cache_kernel}\hspace{0.22cm} & \epsfig{width=0.30\linewidth, file=figures/smp-survey/virtualization}\hspace{0.22cm} & \epsfig{width=0.30\linewidth, file=figures/smp-survey/k42} \\
    (g) Cache-kernel & (h) Virtualization & (i) K42 \\
  \end{tabular}

  \begin{flushleft}
  Figure~\ref{fig:survey:methods}: Multiprocessor operating system
  organizations. Thick lines show locks and the flash symbol denote system
  calls or device interrupts. The figure shows both categories and examples of
  systems.
  \end{flushleft}
\end{figure*}

\section{Categorization}
\label{sec:survey:categorization}
We have categorized the porting approaches into the following implementation
approaches: \emph{giant locking}, \emph{coarse-grained locking},
\emph{fine-grained locking}, \emph{lock-free}, \emph{asymmetric},
\emph{virtualization}, and \emph{reimplementation}. Other surveys and
books~\cite{mukherjee93survey, schimmel94unix} use other categorizations,
e.g., depending on the structuring approach (microkernels and monolithic
kernels). In this section we describe the properties of each of these
implementation methods.


\subsection{Locking-based schemes}

\subsubsection{Giant Locking}
With giant locking (Figure~\ref{fig:survey:methods}a), a single spin lock
protects the entire kernel from concurrent access. The giant lock serializes
all kernel accesses, so most of the uniprocessor semantics can be kept. Giant
locking requires small changes to the kernel apart from acquiring and
releasing the lock, e.g., processor-local pointers to the currently executing
process.  Performance-wise, scalability is limited by having only one
processor executing in the kernel at a time.

Giant locking presents only a minor risk for deadlocks and race conditions
since the kernel access is serialized. The number of places the giant lock
needs to be taken corresponds to the number of entry points into the kernel,
which greatly simplifies the implementation. In terms of porting, the giant
locking approach provides a straightforward way of adding multiprocessor
support since most of the uniprocessor semantics of the kernel can be kept.
However, the kernel also becomes a serialization point, which makes scaling
very difficult for kernel-bound benchmarks. As a foundation for further
improvements, giant locking still provides a viable first step because of its
relative simplicity.

\subsubsection{Coarse-grained Locking}
Coarse-grained locks protect larger collections of data or code, such as
entire kernel subsystems as shown in Figure~\ref{fig:survey:methods}b.
Compared to giant-locking, coarse-grained locks open up for some parallel work
in the kernel.  For example, a coarse-grained kernel can have separate locks
for the filesystem and network subsystems, allowing two processors to
concurrently execute in different subsystems. However, inter-dependencies
between the subsystems can force an effective serialization similar to giant
locking. If subsystems are reasonably self-contained, coarse-grained locking
is fairly straightforward, otherwise complex dependencies might cause data
races or deadlocks.

\subsubsection{Fine-grained Locking}
Fine-grained locking (Figure~\ref{fig:survey:methods}c), restricts the locking
to individual data structures or even parts of data structures.  Fine-grained
locking allows for increased parallelism at the cost of more lock invocations
and more complex engineering. Even fine-grained implementations will sometimes
use coarse-grained locks, which are more beneficial for uncontended data.

Finer granularity of the locks at makes the implementation more prone to
errors such as race conditions and deadlocks, especially compared to the giant
locking approach.

\subsection{Lock-free Approaches}
Using hardware support, it is possible to construct lock-free operating
systems. Lock-free algorithms rely on instructions for atomically checking and
updating a word in memory (compare-and-swap, CAS), found on many CPU
architectures. However, for efficient implementation of lock-free algorithms,
a CAS instruction capable of updating multiple locations is needed, e.g.,
double CAS (DCAS). Simple structures such as stacks and lists can be
implemented directly with CAS and DCAS, while more complex structures use
versioning and retries to detect and handle concurrent
access.~\cite{massalin92lockfree}

A completely lock-free operating system relies on these specialized data
structures. Lock-free data structures are sometimes hard to get correct and
can be inefficient without proper hardware support~\cite{doherty04dcas}, which
limits the scalability of completely lock-free approaches. Also, transforming
an existing kernel to lock-free operation requires a major refactoring of the
kernel internals, so the development costs of a lock-free kernel is likely to
be high.

\subsection{Asymmetric Approaches}
It is also possible to divide the work asymmetrically among the processors.
Asymmetric operating systems assign processors to special uses, e.g., compute
processors or I/O handling processors. Such a system might be beneficial for
cases where there is a high I/O load, or to simplify operating system ports
(as discussed in Section~\ref{sec:survey:asymmetric}).

Since asymmetric systems can be very diverse, both the implementation cost and
scalability of these systems will vary.

\subsection{Virtualization}
A completely different method is to partition the multiprocessor machine into
a virtual cluster, running many OS instances on shared hardware
(Figure~\ref{fig:survey:methods}h). This category can be further
subdivided into fully virtualized and paravirtualized systems, where the
latter employs operating system modifications and virtual extensions to the
architecture to lower virtualization overhead or handle hardware limitations
which makes full virtualization hard to achieve~\cite{robin00intel}.

The virtualizing layer, called a Hypervisor, runs at a higher privilege level
than the operating system kernel. The Hypervisor performs handling and
multiplexing of virtualized resources, which makes the Hypervisor less
demanding to implement than a full operating system kernel. As virtualization
also allows existing uniprocessor operating systems to run with small or no
modifications, the development costs of a port is limited.

As discussed above, not all processor architectures are well suited for full
virtualization. Recent processors therefore have hardware support for
virtualization~\cite{uhlig05virtualization, amd05pacifica}, e.g., through
providing an extra set of privilege levels and trapping on access to
privileged instructions. With this hardware extensions, unmodified operating
systems can run on top of a Hypervisor, although the performance will be
limited by emulation of hardware devices.


\subsection{Reimplementation}
A final approach is to reimplement the core of the kernel for multiprocessor
support and provide API/ABI compatibility with the original kernel.  While
drastic, this can be an alternative for moving to large-scale multiprocessors,
when legacy code might otherwise limit the scalability.

\section{Operating System Implementations}
\label{sec:survey:implementations}
In this section, we discuss different implementations of multiprocessor ports
representing the different porting approaches presented.
Table~\ref{tab:survey:categorization} provides a summary of the discussed
systems.


\subsection{Giant-locking Implementations}

Many multiprocessor ports of uniprocessor operating systems are first
implemented with a giant locking approach, e.g., Linux 2.0~\cite{beck98linux},
FreeBSD~\cite{lehey03freebsd}, and other kernels~\cite{kagstrom05experiences}.
Later releases relaxes the locking scheme with a more fine-grained approach. We discuss
the Linux giant locking more in detail in Section~\ref{sec:survey:linux}.  The
industrial kernel described in~\cite{kagstrom05experiences} uses a giant
locking approach for the first version of the port. The performance of the
port was found to be limited due to a large proportion of in-kernel time. This
port was shown to be difficult due to a large code size and a single-person
development team.

The QNX Neutrino microkernel~\cite{qnx} also protects the kernel with a giant
lock. Latency for QNX is limited by the small amount of code actually executed
within the kernel. However, as most operating system functionality are handled
by server processes, the parallelization of these are more important than the
actual kernel.


\subsection{Coarse-grained locking implementations}

A special case of coarse-grained locking is \emph{funnels} used in DEC
OSF/1~\cite{denham94osf1}, and Mac~OS~X~\cite{gerbag02advanced}. In OSF/1,
code running inside a funnel always executes serialized on a ``master''
processor, similar to master-slave systems.

Mac~OS~X started out with what was effectively a giant lock (a funnel for the
entire BSD portion of the kernel), but thereafter evolved into a more
coarse-grained implementation with separate funnels for the filesystem and
network subsystems. Mac~OS~X does not restrict the funnel to a single
processor. Instead the funnel acts as a subsystem lock, which is released on
thread rescheduling. Currently, Mac~OS~X is reworked to support locking at a
finer granularity.

\begin{table*}[t!]
  \begin{center}
    \caption[Categorized multiprocessor operating systems.]{\emph{Continued on
      next page.}}
    \label{tab:survey:categorization}
    \vspace{0.4in}
    \scriptsize
    \begin{tabular}{l|ll|}
      \cline{2-3}
      System & Method & Focus \\
      \hline
      \hline
      Linux 2.0~\cite{beck98linux}          & Giant            & General purpose      \\
      FreeBSD 4.9~\cite{lehey03freebsd}     & Giant            & General purpose      \\
      QNX~\cite{qnx}                        & Giant            & Real-time            \\
      \hline
      Linux 2.2                             & Coarse           & General purpose      \\
      Mac~OS~X~\cite{gerbag02advanced}      & Coarse           & General purpose      \\
      \hline
      OSF/1~\cite{denham94osf1}             & Fine             & General purpose      \\
      Linux 2.4                             & Fine             & General purpose      \\
      Linux 2.6~\cite{love2003linux}        & Fine             & General purpose      \\
      AIX~\cite{clark95symmetric, talbot95aix} & Fine          & General purpose      \\
      Solaris~\cite{kleinman92solaris}      & Fine             & General purpose      \\
      FreeBSD 5.4                           & Fine             & General purpose      \\
      \hline
      Synthesis~\cite{massalin92lockfree}   & Lock-free        & General purpose      \\
      Cache kernel~\cite{cheriton94caching} & Lock-free        & Application specific \\
      \hline
      Dual VAX 11/780~\cite{goble82dualvax} & Asymmetric       & General purpose      \\
      Application kernel~\cite{ska2004}     & Asymmetric       & Low effort           \\
      Piglet~\cite{muir01piglet}            & Asymmetric       & I/O intensive        \\
      \hline
      Cellular Disco~\cite{govil99cellular} & Virtualized      & Hardware sharing, fault tolerance\\
      VMWare ESX~\cite{rosenblum05virtual}  & Virtualized      & Hardware sharing  \\
      L4Linux~\cite{uhlig04scalablevms}     & Virtualized      & Hardware sharing  \\
      Adeos~\cite{yaghmour2002}             & Virtualized      & Hardware sharing  \\
      Xen~\cite{barham03xen}                & Virtualized      & Hardware sharing  \\
      \hline
      K42~\cite{appavoo03enabling}          & Reimpl.          & Scalability \\
      \hline
      \hline
    \end{tabular}
  \end{center}
\end{table*}

\begin{table*}[t!]
  \begin{center}
    \begin{flushleft}
      Table~\ref{tab:survey:categorization}: Summary of the categorized
      multiprocessor operating systems. The code lines refer to the latest
      version available and the development time is the time between the two
      last major releases.
    \end{flushleft}
    \vspace{0.2in}
    \scriptsize
    \begin{tabular}{l|rr|rl|}
      \cline{2-5}
      & \multicolumn{2}{c}{Performance}\vline & \multicolumn{2}{c}{Effort}\vline \\
      \cline{2-5}
      System & Latency & Scalability & Code lines & Devel. time \\
      \hline
      \hline
      Linux 2.0~\cite{beck98linux}          & High & Low & 955K & 12 months\\
      FreeBSD 4.9~\cite{lehey03freebsd}     & High & Low & 1.9M & ?\\
      QNX~\cite{qnx}                        & Low & ? & ? & ? \\
      \hline
      Linux 2.2                             & Medium & Low & 2.5M & 18 months \\
      Mac~OS~X~\cite{gerbag02advanced}      & High & Low & ? & ? \\
      \hline
      OSF/1~\cite{denham94osf1}             & Low & High & ? & ?\\
      Linux 2.4                             & Medium & Medium & 5.2M & 11 months \\
      Linux 2.6~\cite{love2003linux}        & Low & High & 6.1M & 11 months \\
      AIX~\cite{clark95symmetric, talbot95aix}& Low & High & ? & 18 months \\
      Solaris~\cite{kleinman92solaris}      & Low & High & ? & ? \\
      FreeBSD 5.4                           & Low & High & 2.4M & ? \\
      \hline
      Synthesis~\cite{massalin92lockfree}   & Low & ? & ? & ? \\
      Cache kernel~\cite{cheriton94caching} & Low & ? & 15k & ? \\
      \hline
      Dual VAX 11/780~\cite{goble82dualvax} & High & Low & ? & ? \\
      Application kernel~\cite{ska2004}     & High & Low & 3,600 & 5 weeks \\
      Piglet~\cite{muir01piglet}            & As UP & Dep. on UP & ? & ? \\
      \hline
      Cellular Disco~\cite{govil99cellular} & As Guest & As Guest & 50k & ?\\
      VMWare ESX~\cite{rosenblum05virtual}  & As Guest & As Guest & ? & ? \\
      L4Linux~\cite{uhlig04scalablevms}     & As Guest & As Guest & ? & ? \\
      Adeos~\cite{yaghmour2002}             & As Guest & As Guest & ? & ?\\
      Xen~\cite{barham03xen}                & As Guest & As Guest & 75k+38k & ?\\
      \hline
      K42~\cite{appavoo03enabling}          & Low & High & 50k & ? \\
      \hline
      \hline
    \end{tabular}
  \end{center}
\end{table*}

\subsection{Fine-grained locking implementations}
% FIX: (?) add info about local run queue, work stealing scheduler?
AIX~\cite{clark95symmetric} and DEC OSF/1 3.0~\cite{denham94osf1} were
released with fine-grained locking from the start. In both cases, the SMP port
was based on a preemptible uniprocessor kernel, which simplified porting since
disabling of preemption correspond to places where a lock is needed in the
multiprocessor version.

During the development of OSF/1, funneling was used to protect the different
subsystems while core parts like the scheduler and virtual memory management
were parallelized.  Solaris~\cite{kleinman92solaris} and current versions of
Linux~\cite{love2003linux} and FreeBSD also implement fine-grained locking.


\subsection{Lock-free Implementations}
To our knowledge, there exists only two operating system kernels which rely
solely on lock-free algorithms; Synthesis~\cite{massalin92lockfree} and the
Cache Kernel~\cite{cheriton94caching}. Synthesis uses a traditional monolithic
structure but restricts kernel data structures to a few simple lock-free
implementations of e.g., queues and lists.

The Cache Kernel~\cite{cheriton94caching}
(Figure~\ref{fig:survey:methods}g) provides basic kernel support for
address spaces, threads, and application kernels. Instead of providing full
implementations of these concepts, the Cache Kernel caches a set of active
objects which is installed by the application kernels. For example, the
currently running threads are present as thread objects holding the basic
register state, while an application kernel holds the complete state.  The
Cache Kernel design caters for a very small kernel which can be easily
verified and implemented in a lock-free manner. Note, however, that the
application kernels still need to be parallelized to fully benefit from
multiprocessor operation. Both Synthesis and the Cache Kernel were implemented
for the Motorola 68k CISC architecture, which has architectural support for
DCAS.  Implementations for other architectures which lack DCAS support might
be more difficult.

\subsection{Asymmetric Implementations}
\label{sec:survey:asymmetric}
The most common asymmetric systems have been master-slave
systems~\cite{goble82dualvax}, which employ one master processor to run kernel
code while the other (``slave'') processors only execute user space
applications. The changes to the original OS in a master-slave port mainly
consist of the introduction of separate queues for master and slave jobs, as
shown in Figure~\ref{fig:survey:methods}d. Like giant locks, the performance
of master-slave systems is limited by allowing only one processor in the
kernel.

The Application kernel approach~\cite{ska2004}
(Figure~\ref{fig:survey:methods}e) allows keeping the original
uniprocessor kernel as-is. The approach runs the original unmodified kernel on
one processor, while user-level applications run on a small custom kernel on
the other processors. All processes are divided in two parts, one application
thread and one bootstrap thread. The application threads run the original
application. Kernel interaction by the application threads are handled in the
application kernel, which simply sets a flag for the bootstrap thread. The
bootstrap thread then forwards system calls, page faults etc., to the
uniprocessor kernel, which handles them as before.

Since the application kernel approach requires an extra round-trip for kernel
interaction, the latency of kernel operations increases. The application
kernel approach can still provide good speedup for compute-bound applications
at a low implementation cost.

Piglet~\cite{muir01piglet} (Figure~\ref{fig:survey:methods}f) dedicates the
processors to specific operating system functionality.  Piglet allocates
processors to run a Lightweight Device Kernel (LDK), which normally handles
access to hardware devices but can perform other tasks. The LDK is not
interrupt-driven, but instead polls devices and message buffers for incoming
work. A prototype of Piglet has been implemented to run beside Linux 2.0.30,
where the network subsystem (including device handling) has been offloaded to
the LDK, and the Linux kernel and user-space processes communicate through
lock-free message buffers with the LDK.

% FIX: Add normal disco

\subsection{Virtualization}
There are a number of virtualization systems. VMWare ESX
server~\cite{rosenblum05virtual} is a fully virtualized system which uses
execution-time dynamic binary translation to handle virtualization problems of the
IA-32 platform. Cellular disco~\cite{govil99cellular} is a paravirtualized
system created to use large NUMA machines efficiently. The underlying
Hypervisor is divided into isolated cells, each handling a subset of the
hardware resources to provide fault containment.

Xen~\cite{barham03xen} uses a paravirtualized approach for the Intel IA-32
architecture currently capable of running uniprocessor Linux and NetBSD as
guest operating systems. The paravirtualized approach allows higher
performance than a fully virtualized approach on IA-32.  For example, the real
hardware MMU can be used instead of software lookup in a virtual MMU. The Xen
hypervisor implementation consists of around 75,000 lines of code while the
modifications to Linux 2.6.10, mostly being the addition of a virtual
architecture for Xen, is around 38,000 lines of code. A larger portion of this
is drivers for virtual devices. The Adeos Nanokernel~\cite{yaghmour2002} also
works similar to Xen, requiring modifications to the guest kernel's (Linux)
source code.

As an example of a reimplementation, K42~\cite{appavoo03enabling}
(Figure~\ref{fig:survey:methods}i) is ABI-compatible with Linux but
implemented from scratch.  K42 is a microkernel-based operating system with
most of the operating system functionality executing in user-mode servers or
replaceable libraries. K42 avoids global objects and instead distributes
objects among processors and directs access to the processor-local objects.
Also, K42 supports runtime replacement of object implementations to improve
performance for various workloads.

\section{Linux Case Study}
\label{sec:survey:linux}
% 1.3.28: 17/9 1995 (development)
% 2.0:     2/6 1996   0
% 2.2:    25/1 1999   2.5 years
% 2.4:     4/1 2001   2   years
% 2.6:   17/12 2003   2   years

In order to provide more insight into the trade-offs between performance and
development effort, we have conducted a case study of the evaluation of
multiprocessor support and locking in the Linux kernel.  Linux evolved from
using a giant locking approach in the 2.0 version, through a coarse-grained
approach in 2.2 to using a more fine-grained approach in 2.4 and 2.6.
Multiprocessor support in Linux was introduced in the stable 2.0.1 kernel,
released in June 1996. 18 months later, in late January 1999, 2.2.0 was
released. The 2.4.0 kernel came 11 months later, in early January 2001, while
2.6.0 was released in late December 2003, almost 12 months after the previous
release.

We have studied how three parameters have evolved from kernel versions 2.0 to
2.6. First, we examined the locking characteristics.  Second, we examined the
source code changes for multiprocessor support, and third, we measured
performance for both a kernel-bound benchmark and a compute-bound benchmark.

% Methodology
We chose to compare the latest versions at the time of writing of each of the
stable kernel series, 2.0.40, 2.2.26, 2.4.30, and 2.6.11.7.  We examined files
in \texttt{kernel/}, \texttt{mm/}, \texttt{arch/i386/},
\texttt{include/asm-i386/}, i.e., the kernel core and the IA-32-specific
parts. We also include \texttt{fs/} and \texttt{fs/ext2}. The ext2 filesystem
was chosen since it is available in all compared kernel versions. We exclude
files implementing locks, e.g, \texttt{spinlocks.c}, and generated files.

To see how SMP support changes the source code, we ran the C preprocessor on
the kernel source, with and without \texttt{\_\_SMP\_\_} and
\texttt{CONFIG\_SMP} defined. The preprocessor ran on the file only (without
include-files).  We also removed empty lines and indented the files with the
\textbf{indent} tool to avoid changes in style affecting the results.

\subsection{Evolution of Locking in Linux}
In the 2.0 versions, Linux uses a giant lock, the ``Big Kernel Lock'' (BKL).
Interrupts are also routed to a single CPU, which limits the scalability. On
the other hand, multiprocessor support in Linux 2.0 was possible to implement
without major restructuring of the uniprocessor kernel.

\begin{table}[t]
  \begin{center}
    \caption{Number of locks in the Linux kernel.}
    \label{tab:survey:linux_locks}
    \begin{tabular}{l|rrrrr|r}
      & \multicolumn{5}{c}{Number of locks} \\
      Version   & BKL & spinlock & rwlock & seqlock & rcu & sema\\
      \hline
      2.0.40    &  17 &     0 &   0 &  0 &  0 &  49 \\
      2.2.26    & 226 &   329 & 121 &  0 &  0 & 121 \\
      2.4.30    & 193 &   989 & 300 &  0 &  0 & 332 \\
      2.6.11.7  & 101 & 1,717 & 349 & 56 & 14 & 650 \\
    \end{tabular}
  \end{center}
\end{table}

Linux 2.2 relaxed the giant locking scheme to adopt a coarse-grained scheme.
2.2 also added general-purpose basic spinlocks and spinlocks for
multiple\--readers / single\--writer (rwlocks). The 2.2 kernels has subsystem
locks, e.g., for block device I/O requests, while parts of the kernel are
protected at a finer granularity, e.g., filesystem inode lists and the
runqueue.  However, the 2.2 kernel still uses the giant lock for many operations,
e.g., file read and write.

The 2.4 version of the kernel further relaxes the locking scheme. For example,
the giant lock is no longer held for virtual file system reads and writes.  Like
earlier versions, 2.4 employs a single shared runqueue from which all
processors take jobs.

Many improvements of the multiprocessor support were added in the 2.6 release.
2.6 introduced seqlocks, read-copy update mutual
exclusion~\cite{mckenney02rcu}, processor-local runqueues, and kernel
preemption. Kernel preemption allows processes to be preempted within the
kernel, which reduces latency. A seqlock is a variant of rwlocks that
prioritizes writers over readers. Read-copy update (rcu), finally, is used to
defer updates to a structure until a safe state when all active references to
that structure are removed, which allows for lock-free access. The safe state
is when the process does a voluntary context switch or when the idle loop is
run, after which the updates can proceed.

\subsection{Locking and Source Code Changes}
Table~\ref{tab:survey:linux_locks} shows the how the lock usage has evolved
throughout the Linux development. The table shows the number of places in the
kernel where locks are acquired and released. Semaphores (sema in the table)
are often used to synchronize with user-space, e.g., in the system call
handling, and thus have the same use on uniprocessors.

As the giant lock in 2.0 protects the entire kernel, there are only 17 places
with BKL operations (on system calls, interrupts, and in kernel daemons). The
coarse-grained approach in 2.2 is significantly more complex.  Although 2.2
introduced a number of separate spinlocks, around 30\% (over 250 places) of
the lock operations still handle the giant lock. The use of the giant lock has
been significantly reduced in 2.4, with around 13\% of the lock operations
handle the giant lock. This trend continues into 2.6, where less than 5\% of
the lock operations handled the giant lock. The 2.6 seqlocks and read-copy
update mutual exclusion are still only used in a few places in the kernel.

\begin{table}[t]
  \begin{center}
    \caption{Lines of code with and without SMP support in Linux.}
    \label{tab:survey:linux_smp_changes}
    \begin{tabular}{l|rr|rrr}
      & & & \multicolumn{3}{c}{Lines} \\
      Version & Files & Changed  & No SMP & SMP & Modified/ \\
      & & Files & & & new/ \\
      & & & & & removed \\
      \hline
      2.0.40   &  173 & 22 &  45,392 &  45,770 &   541 \\
      2.2.26   &  226 & 36 &  52,294 &  53,281 & 1,156 \\
      2.4.30   &  280 & 38 &  64,293 &  65,552 & 1,374 \\ %319 & 40 &  82,857 &  84,154 & 1,412
      2.6.11.7 &  548 & 49 & 104,147 & 105,846 & 1,812 \\ %601 & 51 & 132,960 & 134,662 & 1,819 \\
    \end{tabular}
  \end{center}
\end{table}

Table~\ref{tab:survey:linux_smp_changes} shows the results from the C
preprocessor study. From the table, we can see that most files have no
explicit changes for the multiprocessor support. In terms of modified, added,
or removed lines, multiprocessor support for the 2.0 kernel is significantly
less intrusive than the newer kernels, with only 541 source lines (1.19\% of
the uniprocessor source code) modified. In 2.2 and 2.4, around 2.2\% of the
lines differ between the uniprocessor and multiprocessor kernels, while the
implementation is closer again in 2.6 with 1.7\% of the lines changed.


\subsection{Performance Evaluation}
We also did a performance evaluation to compare the different Linux kernel
versions in the Postmark benchmark~\cite{katcherpostmark} and the SPLASH-2 FFT
benchmark\cite{cameron95splash}. The purpose of the performance evaluation is
to show the scalability differences between the versions with two benchmarks
with different characteristics. We use a simulated environment in order to get
access to larger configurations than the hardware we have at hand. The results
from the simulated environment shouldn't be directly compared to actual
hardware, but it should make the scalability properties of the different
kernels clear.

We compiled the 2.0 and 2.2 kernels with GCC 2.7.2 whereas 2.4 and 2.6 were
compiled with GCC 3.3.5. All kernels were compiled with SMP-support enabled,
which is a slight disadvantage on uniprocessors. The system is a minimal
Debian GNU/Linux system which uses the ext2 filesystem which is available in
all kernel versions. We ran 8 Postmark processes in parallel and measured the
time used for all of them to complete.  The benchmark was executed in the
Simics full-system simulator~\cite{simics}, which was configured to simulate
between 1 and 8 processors. Simics simulates a complete computer system
including disks, network, and CPUs including the memory hierarchy modeled
after the Pentium~4.

The Postmark
benchmark models the file system behavior of Internet servers for electronic
mail and web-based commerce, focusing on small-file performance. This
kernel-bound benchmark requires a highly parallelized kernel to exhibit
performance improvements (especially for the file and block I/O subsystems).
We ran Postmark with 10,000 transactions, 1,000 simultaneous files and a file
size of 100 bytes.

The FFT benchmark is a computationally intensive benchmark that performs
Fourier transformations. The benchmark ran with 8 parallel threads (created
with \texttt{pthreads}) for all configurations. We compare the time needed to
complete the entire benchmark.

\begin{figure}[t]
  \begin{center}
    \includegraphics[width=0.70\linewidth]{data/smp-survey/postmark}
  \end{center}
  \caption{Postmark benchmark running on different versions of Linux.}
  \label{fig:survey:postmark}
\end{figure}

\begin{figure}[t]
  \begin{center}
    \includegraphics[width=0.70\linewidth]{data/smp-survey/fft}
  \end{center}
  \caption{SPLASH FFT benchmark running on different versions of Linux.}
  \label{fig:survey:fft}
\end{figure}

Figure~\ref{fig:survey:postmark} presents the scalability results for the
Postmark benchmark while Figure~\ref{fig:survey:fft} shows the scalability of
the SPLASH FFT benchmark. Both results are normalized to uniprocessor
execution time in Linux 2.0.

In the Postmark benchmark we can see that the absolute uniprocessor
performance has increased, with 2.4 having the best performance.  Linux 2.0
and 2.2 do not scale at all with this benchmark, while 2.4 shows some
improvement over uniprocessor mode.  On the other hand, 2.6 scales well up to
8 processors. Since the 2.0 kernel uses the giant locking approach, it is not
surprising that it does not scale in this kernel-bound benchmark.  Since much
of the file subsystem in 2.2 still uses the giant lock and no scalability
improvement is shown. The file subsystem revision in 2.4 gives it a slight
scalability advantage, although it does not scale beyond 3 processors. It is
not until the 2.6 kernel that Linux manages to scale well for the Postmark
benchmark. Linux 2.6 has a very good scalability, practically linear up to 7
processors.

For the compute-bound FFT benchmark, the scalability is similar between all
versions of Linux. An interesting observation is that while the 2.6 kernel
offers better performance on the postmark benchmark, it is slightly behind the
earlier versions for the FFT benchmark. The 2.6 kernel clearly focuses more on
I/O scalability, which appears to be a slight disadvantage for compute-bound
tasks.

\section{Discussion}
\label{sec:survey:discussion}
Figure~\ref{fig:survey:design_space} shows an approximation of the trade-off
between scalability and development effort of the different approaches
presented. It should be noted that when porting a uniprocessor kernel to a
multiprocessor, it is not always possible to freely select the porting
approach. For example, employing the Xen hypervisor is only possible if the
uniprocessor kernel is written for one of the architectures which Xen
supports.

\begin{figure}[t]
  \begin{center}
    \includegraphics[width=0.70\linewidth]{figures/smp-survey/design_space}
  \end{center}
  \caption{The available design space.}
  \label{fig:survey:design_space}
\end{figure}

% Giant locking
The giant locking approach provides a straightforward way of adding
multiprocessor support since most of the uniprocessor semantics of the kernel
can be kept. However, the kernel also becomes a serialization point, which
makes scaling very difficult for kernel-bound benchmarks. As a foundation for
further improvements, giant locking still provides a viable first step because
of its relative simplicity.

% Coarse grained
% Fine-grained
Coarse-grained locking is more complex to introduce than giant locking, as
seen in the Linux case study. Fine-grained locking further adds to the
complexity, but also enables better scalability. The AIX and OSF/1 experiences
indicates that a preemptible uniprocessor kernel simplifies multiprocessor
porting with finer granularity locks.

% Assymetric
Since asymmetric systems are very diverse, both scalability and effort will
vary depending on the approach. In one extreme, the application kernel
provides a generic porting method with low effort at the cost of limited
scalability for kernel-bound applications. Master-slave systems require more
modifications to the original kernel, but have slightly better performance
than the application kernel because of the extra kernel interaction
round-trip. More complex asymmetric systems, such as Piglet, can have good
scalability on I/O-intensive workloads.

% Lock-free
Because of complex algorithms and limited hardware support, completely
lock-free operating systems require high effort to provide good scalability
for ports of existing uniprocessor systems.  Lock-free algorithms are still
used in many lock-based operating systems, e.g., the read-copy update
mechanism in Linux.

% Virtualization
For certain application domains, paravirtualized systems can provide good
scalability at relatively low engineering costs. Virtualization allows the
uniprocessor kernel to be kept unchanged for fully virtualized environments or
with small changes in paravirtualized environments. For example, the Xen
hypervisor implementation is very small compared to Linux, and the changes
needed to port an operating system is fairly limited. However, hardware
support is needed for fault tolerance, and shared-memory applications cannot
be load-balanced across virtual machines.

% Reimplementation
Reimplementation enables the highest scalability improvements but at the
highest effort. Reimplementation should mainly be considered if the original
operating system would be very hard to port with good results, or if the
target hardware is very different from the current platform.


\section{Conclusions}
We have presented a categorization of multiprocessor structuring of operating
systems. We identified seven different approaches, \emph{giant locking},
\emph{coarse-grained locking}, \emph{fine-grained locking}, \emph{lock-free},
\emph{asymmetric}, \emph{virtualization}, and \emph{reimplementation}. These
categories have different performance and implementation complexity
characteristics, with giant locking and asymmetric systems providing the
lowest effort and lowest performance, and reimplementation and fine-grained
locking providing the best performance at a high implementation cost.

The Linux case study illustrates the evolution of multiprocessor support in a
kernel. The 2.0 giant lock implementation was kept close to the uniprocessor.
The implementation then adopted a more coarse-grained locking approach, which
became significantly more complex and also diverged more from the uniprocessor
kernel. The more fine-grained approaches in 2.4 and 2.6 do not increase the
complexity as compared to 2.2, which suggests that the implementation
converges again as it matures.

\section*{Acknowledgments}
This work was partly funded by The Knowledge Foundation in Sweden under a
research grant for the project ``Blekinge - Engineering Software Qualities
(BESQ)'' (\url{http://www.bth.se/~besq}).

\section{Introduction}
A large majority of the mobile phones sold today come with support for the
Java 2 Platform, Micro Edition (J2ME)~\cite{j2me}, and the installation base
can be measured in billions of units~\cite{schwartz06j2mecoverage}. J2ME is a
royalty-free Java development environment for mobile phones, and is often the
only available method of extending the software installed on the phone. This
poses a severe problem when porting C or C++ applications to J2ME devices,
which can often require a complete rewrite in Java of the software, possibly
assisted by automated tools~\cite{martin02ephedra, buddrus98cappucino,
  malabarba99mohca, jazillan}.

Cibyl is a binary translator which targets this problem. Cibyl translates MIPS
binaries into Java bytecode, and provides means to integrate with native J2ME
classes. Cibyl therefore allows C and C++ programs to be ported to run on J2ME
phones. When designing Cibyl, our goals have been to produce compact
translated code with performance close to native Java code for common cases.
The general design of Cibyl has been described in an earlier
paper~\cite{kagstrom07cibyl}, and this paper focuses on optimizations made
to reduce the size and improve the performance of the translated binaries.

The main contributions of this paper are the following. We first describe the
set of optimizations we make and how these improve size and performance. We
then perform a benchmark on an application ported with Cibyl to illustrate the
optimizations in a real-world setting. Finally, we compare Cibyl against
native Java and another binary translator, NestedVM~\cite{alliet04nestedvm} in
a micro benchmark (an implementation of the A* algorithm) to study performance
characteristics in detail. The performance results show that Cibyl is
significantly faster than NestedVM and close to native Java performance on the
cases we target.

The rest of the paper is structured as follows. In Section~\ref{sec:cp:cibyl}
we introduce the Cibyl binary translator. The main part of the paper then
follows in Section~\ref{sec:cp:optimizations} where we describe the
optimizations performed by Cibyl and Section~\ref{sec:cp:evaluation} where we
evaluate our optimizations. Section~\ref{sec:cp:related_work} describes
related work and finally we conclude and present future directions in
Section~\ref{sec:cp:conclusions}.

\section{Cibyl}
\label{sec:cp:cibyl}
\begin{figure*}[t]
  \begin{center}
    \includegraphics[width=0.96\linewidth]{figures/cibyl-performance/compilation}
    \caption[Cibyl translation process]{Translation process in Cibyl. Gray boxes show unmodified
      third-party tools.}
    \label{fig:cp:compilation}
  \end{center}
\end{figure*}

Cibyl uses the GCC~\cite{gcc} compiler to produce a MIPS~I~\cite{kane88mips}
binary, which is thereafter translated into Java bytecode by the Cibyl tools.
Figure~\ref{fig:cp:compilation} shows the translation process with Cibyl,
where we use a set of tools to translate the MIPS binary. Apart from the
binary translator, which outputs Java bytecode assembly to
Jasmin~\cite{jasmin}, we also have a stub code generator to produce stubs for
calls into native Java code. When translating, Cibyl uses Java local variables
to represent MIPS registers, which contributes to producing efficient and
compact code compared to using class member variables or static class
variables. The MIPS instruction set is well suited for binary translation, and
most arithmetic instructions can be directly translated to a sequence of
loading source registers (local variables) on the Java operand stack,
performing the operation and storing into the destination register.

We use a Java integer array to represent memory as seen by the MIPS binary.
This means that 32-bit memory accesses are performed efficiently by simply
indexing the memory array with the address divided by four, but also that 8-
and 16-bit accesses need an extra step of masking and shifting the value in
memory to get the correct result. Since a common pattern is to use the same
base register repeatedly with different offsets, we pre-calculate the array
index and use special memory access registers for these cases. To reduce space, we
also perform the more expensive 8- and 16-bit accesses through functions in
the runtime support instead of generating bytecode directly. Similarly,
expensive arithmetic operations such as unsigned multiplications are also
implemented in the runtime layer. Since 32-bit access is easiest to support
efficiently, Cibyl focuses on performance for this case.

Cibyl uses a 1-1 mapping between C functions and generated Java methods, which
brings a number of benefits. First, this mapping enables the J2ME profiler to
produce meaningful output for Cibyl programs. Second, if the program causes an
exception, the call chain emitted by the JVM will be human readable. The 1-1
mapping also enables some optimizations, which will be discussed later.  We
handle register-indirect function calls specially since Java bytecode does not
support function pointers. To support function pointers, we generate a special
``call table'' method that switches on the function address and calls the
corresponding method indirectly.

Compared to the integer instruction set, the MIPS floating point instruction
set is more difficult to translate~\cite{kagstrom07cibyl}. Floating point is
therefore supported by a hybrid approach where we use the GCC soft-float
support, but implement the runtime support functions using native Java floats.
This solution provides a tradeoff between implementation complexity and
performance, with a very simple implementation but less performance than an
implementation of the MIPS FPU instruction set.

\section{Optimizations}
\label{sec:cp:optimizations}
We perform a number of optimizations in Cibyl apart from the general code
generation optimizations described above to improve performance and reduce the
size of the generated Java class files.

\subsection{32-bit multiplications/divisions}
The MIPS instruction for multiplication always produce a 64-bit result, split
in the \texttt{hi}/\texttt{lo} register pair. We translate this to Java
bytecode by casting the source registers to 64-bit longs, perform the
multiplication, split the resulting value and place it in
\texttt{hi}/\texttt{lo}. As expected, this generates many instructions in Java
bytecode, and is also fairly inefficient and we perform it in the runtime
support to save space.

Often, however, only the low 32 bits of the value is actually used and in
these cases we perform a 32-bit multiplication which can be done natively in
Java bytecode. This is safe since the \texttt{lo}/\texttt{hi} registers are
not communicated across functions in the MIPS ABI, so if the \texttt{hi}
register is unused we simply skip producing it. This optimization saves both
space and improves performance of multiplications. Divisions are handled
similarly.

\subsection{Size reduction}
To reduce size, functions which are unused are pruned during translation, and
relocation information in the binary is used to determine which functions are
safe to remove. Similarly, the table of indirect calls contains only the
functions which are called register-indirect, also determined by relocations
and reconstructing values from instructions with relocations.

The 1-1 mapping between C functions and Java also methods allows for size
reductions together with the MIPS ABI~\cite{sco96mipsabi}. On function calls,
we pass only the stack pointer and the argument registers used by the called
function, which reduces the overhead for short functions. Another optimization
the ABI allows is to skip stores and loads in the function prologue and
epilogue to the MIPS callee-saved registers \texttt{s0}...\texttt{s8}, which
are handled automatically as the register representation uses Java local
variables.

%We use execution profiles to arrange the output into multiple classes where we
%co-locate methods based on execution count to co-locate ``hot'' methods in the
%%same class. A special case is methods which are never executed in the profile,
%and are placed in separate classes. Although this optimization often increases
%the JAR size, placing unused methods into separate classes saves device memory
%since the JVM typically loads classes on demand.

\begin{figure*}[thb]
  \begin{center}
    \includegraphics[width=0.99\linewidth]{figures/cibyl-performance/colocated}
    \caption{Handling of co-located functions in Cibyl}
    \label{fig:cp:colocated}
  \end{center}
\end{figure*}

\subsection{Inlining of builtin functionality}
We perform selective inlining of certain functions in the runtime support,
mostly for the floating point support. This optimization is implemented by
matching function call names to a set of Python classes that implements these
builtins. These then generate the corresponding functionality through
emitting bytecode instructions directly, replacing the function call.

This allows us to reduce the overhead of floating point operations
significantly, at the cost of a slightly larger translated file. We use this
functionality for other purposes as well, e.g., to throw and catch Java
exceptions from C code.

\subsection{Function co-location}
One large source of overhead is method calls, i.e., translated C function
calls. Since Cibyl uses local variables for the register representation, it
needs to pass up to 7 arguments for the method to call, and method call
overhead grows with the number of arguments. This overhead is especially
noticeable with short functions which are frequently called.

As a way around this problem, we allow multiple C functions to be co-located
into one Java method. Calling between functions in a single Java method can
then be done using regular \texttt{goto}'s avoiding the method call overhead.
The implementation is illustrated in Figure~\ref{fig:cp:colocated}, which
shows a call chain \textbf{fn1}, \textbf{fn2}, \textbf{fn3}:

\begin{enumerate}
\item Co-located methods are called with an index specifying the function to
  call, then the method prologue does a switch on this index and calls the
  function. The MIPS \texttt{ra} register is set to a special value to signify
  that the function was called from outside the method.
\item Calls to external methods are handled as elsewhere, with argument
  registers (Java local variables) passed.
\item On calls to functions within the method, a direct \texttt{goto} is used.
  Passing argument registers are not needed, but we store a generated index
  for the return address in the \texttt{ra} register (local variable).
\item On returning, we switch on the \texttt{ra} register and jump back to
  where the local call was made or out of the co-located method.
\end{enumerate}

There are a few differences compared to normal methods. First of all we need
to save the \texttt{ra} register since it's now used for the function
returns. Second, we allocate the used MIPS \texttt{s}-registers to different
Java local variables for each function in the co-located method, which allows
us to avoid storing store/restore these registers in the function as with
normal methods.

\subsection{Peephole optimizer}
Cibyl also includes a peephole optimizer to improve common translation
patterns. For example, since the size of MIPS instructions is fixed at 4
bytes, constant assignments to registers is split in a
\texttt{lui}/\texttt{addiu} pair to assign the upper and lower half of the
register. These assignments are fairly common and are coalesced into one
assignment by the peephole optimizer. Similarly, storing of intermediate
results for computations in registers can often be avoided and kept on the
Java operand stack.

\section{Performance evaluation}
\label{sec:cp:evaluation}
We have evaluated Cibyl performance in two benchmarks, FreeMap and A*.
FreeMap~\cite{shabtai07roadmap} is a GPS street navigation application
originally written for X-windows but later ported to other platforms such as
WindowsCE. It has been ported to the J2ME platform using Cibyl by an external
developer, and consists of over 40000 lines of C code and 1600 lines of Java
code. FreeMap uses a mixture of integer and floating point operations, and
displaying the map is the most computationally intensive operation.

We run FreeMap in the Sun J2ME emulator and set it up to perform a map
rotation operation during one minute and count the number of iterations of the
main loop during this time. Startup and shutdown time is ignored. We compare a
non-optimized Cibyl version with an optimized one, where the optimizations
enabled are inlining of builtins, optimization of 32-bit multiplications and
divisions and co-location of functions dealing with redrawing.

The A* benchmark consists of two implementations of the A* algorithm, one in C
and one in Java, which are based on the same source. The Java implementation
is not a port, but implemented Java-style using multiple classes. Both
implementations stress memory management, dereferencing pointers/references
and short function calls. The graph search visits 35004 nodes during the
execution.

\begin{figure}[bth]
  \begin{center}
    \includegraphics[width=0.58\linewidth]{data/cibyl-performance/freemap}
    \caption[FreeMap benchmark results]{FreeMap benchmark results. The baseline shows results without
      optimizations enabled.}
    \label{fig:cp:freemap}
  \end{center}
\end{figure}

We setup the A* implementation to use different data types for the main data
structure (the nodes in the graph). The types are 32-bit \texttt{int}, 16-bit
\texttt{short}, 32-bit \texttt{float} and the 64-bit \texttt{double}. We run
the benchmark in Cibyl both without and with optimizations, in NestedVM and in
Java on the Sun JDK 1.5.0~\cite{sunjdk} JVM, the Kaffe JVM~\cite{kaffe}, the
SableVM~\cite{gagon03sablevm} interpreter and the GNU Java bytecode
interpreter (gij)~\cite{gcj}. The Cibyl optimizations we use is inlining of
builtins, memory registers, multiplication and division optimization, and function
co-location. We co-locate the functions in the hottest call chain, which is
the actual A* algorithm, looking up nodes, iterating over nodes and computing
distance heuristics.

All benchmarks are executed on a Pentium~M processor at 600 MHz with Debian
GNU/Linux. The A* benchmarks are compiled with GCC 3.3.6 (using the NestedVM
compiler), using the default options for NestedVM and Cibyl, optimizing at
level -O3 in the NestedVM case and -Os, size, for Cibyl. The FreeMap
benchmark is compiled with GCC 4.1.2, optimizing for size.

\subsection{Results}
For FreeMap, shown in Figure~\ref{fig:cp:freemap}, we see that enabling the
optimizations improves the performance with almost 15\% over the non-optimized
version. This improvement is visible in the emulator update frequency.  The
majority of the improvement comes from function co-location and the use of
builtins. The size of the FreeMap classes is 592KB of which 457KB are
Cibyl-generated classes.

% No opt:          3497
% Only O:          3648
% all:             4046
% builtins and O:  3961
% coloc only:      3981
\begin{figure*}[t]
  \begin{center}
    Integer

    \includegraphics[width=0.45\linewidth]{data/cibyl-performance/int-sun}~\includegraphics[width=0.45\linewidth]{data/cibyl-performance/int-kaffe}\\

    Short

    \includegraphics[width=0.45\linewidth]{data/cibyl-performance/short-sun}~\includegraphics[width=0.45\linewidth]{data/cibyl-performance/short-kaffe}\\

    Float

    \includegraphics[width=0.45\linewidth]{data/cibyl-performance/float-sun}~\includegraphics[width=0.45\linewidth]{data/cibyl-performance/float-kaffe}\\

    Double

    \includegraphics[width=0.45\linewidth]{data/cibyl-performance/double-sun}~\includegraphics[width=0.45\linewidth]{data/cibyl-performance/double-kaffe}\\
    \caption[A* benchmark results]{Results of the A* benchmark for the integer,
    short, float and double data types \emph{continued on the next page}}
    \label{fig:cp:astar}
  \end{center}
\end{figure*}

\begin{figure*}[t]
  \begin{center}
    Integer

    \includegraphics[width=0.45\linewidth]{data/cibyl-performance/int-sablevm}~\includegraphics[width=0.45\linewidth]{data/cibyl-performance/int-gij}\\

    Short

    \includegraphics[width=0.45\linewidth]{data/cibyl-performance/short-sablevm}~\includegraphics[width=0.45\linewidth]{data/cibyl-performance/short-gij}\\

    Float

    \includegraphics[width=0.45\linewidth]{data/cibyl-performance/float-sablevm}~\includegraphics[width=0.45\linewidth]{data/cibyl-performance/float-gij}\\

    Double

    \includegraphics[width=0.45\linewidth]{data/cibyl-performance/double-sablevm}~\includegraphics[width=0.45\linewidth]{data/cibyl-performance/double-gij}\\
  \begin{flushleft}
    Figure~\ref{fig:cp:astar}: Results of the A* benchmark for the integer,
    short, float and double data types
  \end{flushleft}
  \end{center}
\end{figure*}

In the A* benchmark, presented in Figure~\ref{fig:cp:astar}, we can see that
Cibyl performs very well in the integer-case, being on par with Java on the
Sun JVM and significantly faster than NestedVM on all tested JVMs. This is
expected since our optimizations target the of 32-bit data case, and the
optimizations improve results with between 10-40\% depending on JVM.  Cibyl is
faster than NestedVM also in the unoptimized case, which is most likely caused
by the higher use of Java local variables in Cibyl (which are more efficient
to access than class members). NestedVM also references memory in a two-level
scheme to support sparse address spaces~\cite{alliet04nestedvm}, which
contributes a bit to the overhead.
\FloatBarrier

For the other cases, there is a more mixed picture. As expected, both Cibyl
and NestedVM are far behind Java in these cases since the translated code
cannot work directly with 16-bit or 64-bit values and less efficiently with
floating point values. With shorts, Cibyl performs better than NestedVM on the
Sun JVM and SableVM and marginally worse on Kaffe and Gij. The slowdown
compared to the integer benchmark is caused by additional memory latency, and
the relative slowdown compared to NestedVM can be explained by the calls into
the runtime for 16-bit memory accesses.

The floating point part, both Cibyl and NestedVM frequently need to store
floats in integer variables or memory (through the method
\texttt{Float.\-floatBitsToInt}), which decrease the performance a lot compared
to native Java. For the Sun JVM, Cibyl has performance comparable to NestedVM,
but is behind on the other JVMs, which is caused by soft-float overhead.
However, we can see that the builtin optimization substantially improves
performance with 30-40\%. The double case is not optimized by the builtin
approach, and show only small improvements from the optimization.

An outlier is native Java on the Kaffe JVM, which shows much worse results
than on the other JVMs throughout all the tests. On the other hand, Kaffe
gives good results with Cibyl and NestedVM, which use a simpler program
structure based on calls of static methods.

The size of the classes including runtime support in A* benchmark is 37KB for
the optimized Cibyl version, 240KB for NestedVM and 61KB for native Java.
Cibyl loads static data (the \texttt{.data} and \texttt{.rodata} ELF sections)
from a file, and with that included in the class size as done in NestedVM and
native Java, the Cibyl size is 75KB. The size advantage compared to NestedVM
is caused by the larger NestedVM runtime, and the more aggressive use of Java
local variables in Cibyl which decreases the size compared to class members.

\FloatBarrier
\section{Related work}
\label{sec:cp:related_work}
NestedVM~\cite{alliet04nestedvm} has many similarities with Cibyl. NestedVM is
also a binary translator that translates MIPS binaries into Java bytecode, but
NestedVM targets security - being able to run insecure binaries in a secure VM
- while Cibyl targets portability to J2ME devices. This is reflected in some
of the design decisions, where NestedVM uses a two-level scheme for memory
accesses (but which can be disabled) to detect accesses of uninitialized
memory, and an optional UNIX compatibility layer to support translation of
existing UNIX tools. Cibyl on the other hand focuses on generation of compact
code and good performance for the common and easily supported case of 32-bit
memory accesses. Cibyl also uses Java local variables for the register
representation throughout, whereas NestedVM uses local variables only for
caching the normal class variable register representation.

There is also a set of compilers which can generate Java bytecode directly.
Axiomatic solutions~\cite{axiomsol} has a compiler for a subset of C which
generates Java bytecode. Cibyl supports the C language fully, and with runtime
support any language which GCC can compile. The University of Queensland
Binary Translator project has a Java bytecode backend for
GCC~\cite{cifuentes00UAB}. This backend is not part of mainline GCC, and
tracking mainline development can require a significant effort. In contrast,
Cibyl is independent of GCC and able to any GCC version which can generate
MIPS binaries.

\section{Conclusions and future work}
\label{sec:cp:conclusions}
In this paper, we have presented the optimization framework for the Cibyl
binary translator and benchmarked it against NestedVM and native Java. We show
how function co-location, inlining of the soft-float implementation and use of
the MIPS ABI contributes to improve the performance of code translated with
Cibyl. Our benchmarks illustrates how these optimizations can improve
performance of real-world Cibyl applications and how binary translation is
affected by data types. We also compare Cibyl to the NestedVM binary
translator and native Java and show that performance in the case we target is
significantly better than NestedVM and close to performance of native Java.

Future directions to improve performance includes implementing the MIPS FPU
instruction set, which is an area where NestedVM has an advantage. We are also
planning an implementation of register type and value tracking, which can
reduce size and improve performance by avoiding arithmetic operations with
constant values and by reducing conversions between floats and integer
representations of floating point values.

\section*{Acknowledgements and Availability}
This work was partly funded by The Knowledge Foundation in Sweden under a
research grant for the project ``Blekinge - Engineering Software Qualities
(BESQ)'' (\url{http://www.bth.se/~besq}). Cibyl is free software under the GNU
GPL and can be downloaded from \url{http://cibyl.googlecode.com}.

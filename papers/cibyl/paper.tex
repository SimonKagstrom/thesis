\section{Introduction}
The Java 2 Platform, Microedition (J2ME)~\cite{j2me} has become practically
ubiquitous among mobile phones with an estimated installation base of around 1
billion units~\cite{schwartz06j2mecoverage}. J2ME provides a royalty-free
development environment where it is possible to extend the capabilities of
mobile phones and other embedded systems through Java.  J2ME is often the
\emph{only} openly available environment for extending mobile phones, and
developers writing software to J2ME-capable embedded devices are therefore
locked to the Java language. When porting existing software written in
languages such as C or C++ to J2ME devices, the development environment can
require a complete rewrite of the software package. Developers are then faced
with either porting their code to another language, or use automated
tools~\cite{martin02ephedra, buddrus98cappucino, malabarba99mohca, jazillan}
which may generate code which is difficult to modify, require manual fixes and
can sometimes be inefficient. Even when implementing new projects for J2ME,
Java might not always be the preferred language. For example, developer
language experience, personal preferences, availability of existing libraries
or co-development for other targets might favor new implementations in other
languages.

In this paper, we present the Cibyl programming environment which allows
existing code written in C and other languages to be recompiled as-is or with
small modifications into Java bytecode and run on J2ME devices. Performance of
the recompiled code can be close to native Java implementations and with
modest space overhead. In contrast to other approaches~\cite{axiomsol}, Cibyl
supports the full C language, and support for C++ and other languages
require only library extensions. Cibyl is not a compiler, but instead relies
on the GCC~\cite{gcc} compiler to produce a MIPS binary. Cibyl does a static
binary translation of a MIPS executable into Java bytecode, and provides a
runtime library to support execution in the Java environment. Compared to
writing a backend for GCC which directly generates Java bytecode, the Cibyl
approach allows for a lower initial effort and also removes the burden of
long-time maintenance of the backend. Using unmodified standard tools also
means that it automatically benefits from tool improvements.

The main contributions of the paper are the following. First, we show how C
programs can be recompiled into Java bytecode and identify problematic areas.
Second, we show that knowledge about the compiled code and the ABI
(Application Binary Interface) can be utilized to generate more efficient
bytecode.  Third, we illustrate how extensions to the MIPS architecture can be
used to provide efficient calls to native Java methods.

The rest of the paper is structured as follows. Section~\ref{sec:cibyl:technology}
describes the technology used in Cibyl. Section~\ref{sec:cibyl:evaluation} presents
an evaluation of the generated code in terms of performance and size.
Thereafter, Section~\ref{sec:cibyl:related_work} describes related work, and finally
Section~\ref{sec:cibyl:conclusions} presents conclusions and future research.

\section{Technology}
\label{sec:cibyl:technology}
Cibyl targets the MIPS~I~\cite{kane88mips}, only using instructions available
in user-space. Compared to many other architectures, MIPS provides a number of
advantages for efficient binary translation. First, regular loads and stores
are always done on aligned addresses, which simplifies memory handling in
Java.  Second, MIPS uses the general-purpose register set for almost all
operations and does not have implicitly updated flag registers, which allows a
straightforward translation of most arithmetic instructions.  Third, MIPS only
does partial register updates for the seldom used unaligned memory accesses
instructions.

\begin{figure*}[htb]
  \begin{center}
    \includegraphics[width=0.99\linewidth]{figures/cibyl/compilation}
    \caption[Cibyl translation process]{The compilation process. Gray boxes show third-party tools and
      white boxes are implemented in Cibyl.}
    \label{fig:cibyl:compilation}
  \end{center}
\end{figure*}

To achieve good performance of the translated binaries, we place a number of
soft restrictions on the generated code and add extensions to the
architecture. In particular we focus on good performance of 32-bit memory
accesses and operations on signed 32-bit values, which are easier to support
efficiently since Java has no unsigned types. We have also made use of
extensions to the MIPS ISA, which is possible since the generated code targets
a virtual machine and does not need to run on actual hardware.

Cibyl builds on the GNU toolchain~\cite{gcc}, which we use to produce the MIPS
binaries in a translation-friendly format. GCC is used to compile the C
source for the MIPS~I instruction set, which is thereafter linked using GNU
ld. We use GCC and ld options to simplify certain operations.  For example, we
turn off the generation of explicit checks for integer division by zero, which
is not needed in Java bytecode where the instruction throws a divide-by-zero
exception. Further, we always work on static executables and therefore disable the
generation of position-independent code. The data and read-only data sections
from the ELF binary is placed in a file which the runtime system loads into
memory on startup. Cibyl uses five steps to compile C source code into a J2ME
JAR-file, illustrated in Figure~\ref{fig:cibyl:compilation}:

\begin{enumerate}
\item The C source is compiled and linked with GCC using the Cibyl headers and
  libraries.
\item The API to Java/J2ME (defined in a C header-file) and the compiled
  program is passed to another tool that generates a Java source file
  containing wrappers for system call stubs.  The set of system calls used by
  a program is known at compile time by feeding back the compiled program to
  the tool and only needed stubs are generated.
\item The \texttt{cibyl-mips2java} tool recompiles the linked program from
  step 2 into Java assembly.
\item The Jasmin~\cite{jasmin} assembler compiles the Java assembly into a class
  file
\item The regular Java compiler compiles the generated system call wrappers and
  runtime support files
\item Finally, the compiled class-files are preverified and combined by the
  Java archiver to a downloadable JAR file. The preverification step is needed
  for J2ME programs since the verification capabilities of the mobile JVM is
  limited.
\end{enumerate}


\subsection{Memory Access}
\label{sec:cibyl:cibyl:memory}
We use a linker script to link the text segment high up in the memory and the
initialized and uninitialized data starting at address zero. The translated
text segment cannot be addressed, and is therefore not loaded into memory.
Figure~\ref{fig:cibyl:address_space} shows the address space in Cibyl. During
startup, a configurable portion of the Java heap is allocated to the Cibyl
program. The stack pointer is setup to the end of the address space, and the
heap starts after the uninitialized data. The heap manager is a standard
\emph{malloc}/\emph{free} implementation written in C.

\begin{figure*}[thb]
  \begin{center}
    \includegraphics[width=0.80\linewidth]{figures/cibyl/cibyl-address-space}
    \caption[Cibyl address space]{Cibyl address space. The end of the address space depends on the
      available memory}
    \label{fig:cibyl:address_space}
  \end{center}
\end{figure*}

We strive to provide efficient 32-bit memory accesses while accepting a
performance cost for 8- and 16-bit accesses. Memory is therefore represented
as an integer-vector, which means that 32-bit loads and stores can be
performed by indexing this vector. As the mapped data starts at address 0 and
is contiguous from there, the computed address can be used directly as an
index after right-shifting it by 2. Figure~\ref{fig:cibyl:memory_registers}
shows translation of memory accesses.

To further improve the performance of 32-bit memory accesses, we allocate
extra registers to optimize multiple memory accesses where the base address
register stays the same.  The key is that since the base address is constant,
the right-shift performed to translate the address into a Java vector index
need only be done once. The analysis is done on basic blocks, and replaces
registers if there are more than two accesses to a constant address (shown in
the right part of Figure~\ref{fig:cibyl:memory_registers}). With these
optimizations, each 32-bit memory access can be done with between 4 and 8 Java
bytecode instructions.

\begin{figure*}[thb]
  \begin{center}
    \includegraphics[width=0.80\linewidth]{figures/cibyl/memregs}
    \caption[Cibyl memory access]{Cibyl memory accesses. The left part shows normal memory accesses
      and the right part shows memory accesses using the special memory
      registers.}
    \label{fig:cibyl:memory_registers}
  \end{center}
\end{figure*}

8- and 16-bit memory loads and stores also operate on the memory
integer-vector, but require more work. For example, a store byte operation
must first load the 32-bit word from the memory vector, mask out the requested
byte, shift the value to be stored to the correct byte address and perform a
bitwise \emph{or} to update the memory location. Signed loads (with the MIPS
\texttt{lb} and \texttt{lh} instructions) also need sign-extension. 8- and
16-bit accesses generate between 20-42 bytecode instructions, depending on the
sign and size.  To save space, these accesses are performed through functions
in the runtime support.

\subsection{Code Generation}
The MIPS binaries are translated by the \texttt{cibyl-mips2java} tool to Java
bytecode assembly, which is thereafter assembled into bytecode by the Jasmin
assembler~\cite{jasmin}. The tool produces one class, which is split up in one
method per C function for the recompiled code. During parsing, \texttt{nop}
instructions and unused functions are discarded and instructions in delay
slots are appended to the branch instruction.

We use local Java variables to store registers to improve JVM
optimization~\cite{budimlic97optimizing} and produce more compact bytecode.
The MIPS \texttt{hi} and \texttt{lo} registers, which are used to store
results from multiplications and divisions, are stored in static variables
since these must sometimes be performed in the runtime support. Normal
arithmetic instructions require between 2 and 4 bytecode instructions, but we
simplify cases where Java bytecode instructions permit a more efficient
translation (e.g., \texttt{addi} when used to increase a register by a
constant).

% delay slot handling like in NestedVM~\cite{alliet04nestedvm}

We do a number of optimizations on the recompiled code. First, by retaining
the relocation information in the MIPS executable, we are able to produce a
smaller output binary. The relocation information allows discarding of unused
functions (functions which have no entries in the relocation tables), which is
not done by the GNU linker. Second, we use architectural extensions to make
calls to native Java functionality efficient, which is described in more
detail in Section~\ref{sec:cibyl:native_java}. Third, by employing knowledge of the
MIPS ABI~\cite{sco96mipsabi} and the program structure, we are able to
translate problematic instructions more efficiently.

There are four groups of MIPS instructions that are problematic to translate:
instructions for unaligned memory access, instructions dealing with unsigned
values, 8- and 16-bit load/stores, and multiplication and division.  Unaligned
memory access is uncommon in most programs since it is inefficient also on
native hardware. We therefore handle these instructions by invoking methods in
the runtime environment. Operations on unsigned values are also handled in the
runtime where needed, e.g., for multiplications.

Multiplication is problematic because the MIPS \texttt{mult} instruction
generates a 64-bit result split between the special \texttt{hi}/\texttt{lo}
registers. In Java, translating \texttt{mult} means promotion of the argument
to the type \texttt{long}, performing a 64-bit multiplication,
right-shifting the high part of the result into a temporary and then
converting both results back to integers and storing in \texttt{hi} and
\texttt{lo}. The runtime support for division and multiplication require 9-38
bytecode instructions. For the common case of signed operations where only the
low 32-bits are used, we can most of the time perform the operation in the
same way as other arithmetic instructions. We do this by omitting the
computation of the \texttt{hi} value for functions which only read the
\texttt{lo} value. This can be done because the ABI specifies that function
results are never passed in the \texttt{hi} or \texttt{lo} registers.

\begin{figure*}[htb]
  \centering
  \scriptsize
\begin{verbatim}
typedef union {
   float f;  uint32_t i;
} float_union_t;

float __addsf3(float _a, float _b) {      public static int __addsf3_helper(int _a,
  float_union_t a, b, res;                                                  int _b) {
                                            float a = Float.intBitsToFloat(_a);
  a.f = _a; b.f = _b;                       float b = Float.intBitsToFloat(_b);
  res.i = __addsf3_helper(a.i,b.i);
                                            return Float.floatToIntBits(a + b);
  return res.f;                           }
}
\end{verbatim}
  \caption[Cibyl floating point support]{Cibyl floating point support. The left part of the figure shows the
    C runtime support, the right part shows the Java implementation of the
    operation}
  \label{fig:floating_point}
\end{figure*}

A custom peephole optimizer runs on the translated code to remove optimize
some inefficiencies in the generated bytecode. This primarily helps with
removing extra stores to registers (Java local variables), which are needed in
MIPS code but which can be kept on the Java computation stack in Java
bytecode.

\subsection{Floating point support}
Cibyl supports floating point, but does not implement translation of the MIPS
floating point unit instructions. Compared to the general-purpose instruction
set, the floating point instructions are more difficult to translate
efficiently. Many of the floating point instructions have side effects on
status registers, and while this can often can be handled lazily as done in
FX!32~\cite{hookway97fx32}, it complicates the implementation. A further
problem is that double-precision operations use pairs of single-precision
registers, which makes it difficult to store registers in Java local variables
and instead requires expensive conversion.

Cibyl supports floating point operations through a hybrid approach where we
utilize the GCC software floating point support, but implement it using Java
floating point operations, utilizing hardware support where available.
Figure~\ref{fig:floating_point} illustrates how the floating point support
works in Cibyl. When compiling for softfloats, GCC generates calls to runtime
support functions for floating point operations, e.g., \texttt{\_\_addsf3} to
add two \texttt{float}'s. The Cibyl implementation of \texttt{\_\_addsf3} is
shown on the left part of Figure~\ref{fig:floating_point}, and is simply a
call to a Java helper function which is shown on the right part of the figure.
The Java implementation will parse the integer pattern as a floating point
number (usually just a move to a floating point register), perform the
operation and return the resulting integer bit-pattern. This structure
requires only runtime support and no changes to the binary translator.

\subsection{Function calls}
\label{sec:cibyl:function_calls}
MIPS has two instructions for making procedure calls, \texttt{jal} which calls
a statically known target address, and \texttt{jalr} which makes a
register-indirect call. Both these instructions store the return address in
the \texttt{ra} register. Cibyl maps C functions to static Java methods and
makes use of the MIPS ABI~\cite{sco96mipsabi} to provide better performance
and smaller size of the generated code. We disable the GCC optimization of
tail calls so that all calls are either done through \texttt{jal} or
\texttt{jalr} instructions. For statically known call targets, Cibyl will then
generate a normal Java call to a static method.  As an optimization, we only
pass registers which are actually used by the function and likewise only
return values from functions that modify return registers in the ABI.

\begin{figure*}[htb]
  \begin{center}
    \includegraphics[width=0.82\linewidth]{figures/cibyl/cross-method-call}
    \caption{Handling of indirect function calls in Cibyl}
    \label{fig:cibyl:function_calls}
  \end{center}
\end{figure*}

Keeping the register state in local variables and a one-to-one mapping of C
functions to Java methods provides some benefits. Of the 32 general-purpose
MIPS registers, only at most seven are transfer state between functions with
the MIPS ABI. These are the stack pointer \texttt{sp}, the four argument
registers \texttt{a0}--\texttt{a3} and the two registers for return values.
Other registers are either free to use by the target function or must be
preserved. The storing and restoring of the preserved registers in the
function prologue and epilogue can be optimized away since each Java method
has a private local variable scope.

Since Java bytecode does not allow calls to computed targets, we handle the
\texttt{jalr} instruction differently. The register-indirect calls are handled
through passing the address to a generated static method which looks up the
target function address in a lookup table and invokes the corresponding
function. While MIPS branch instructions with statically known addresses have
corresponding Java bytecode instructions, register-indirect branches (used by
GCC for example to optimize switch-statements) pose the same problem as the
\texttt{jalr} instruction. We also solve this problem in the same way, by a
method-local lookup table with possible branch targets in the function. The
binary translator use the relocation information and scans the data segment
for possible branch targets within the function.

Figure~\ref{fig:cibyl:function_calls} illustrates how indirect function calls
work in Cibyl. The code involves two functions, \texttt{main} and
\texttt{printf} where \texttt{main} calls \texttt{printf} through the
register-indirect \texttt{jalr} instruction. The indirection is handled via
the special \texttt{globaltab} method for indirect calls.

\subsection{Calls to Native Java Methods}
\label{sec:cibyl:native_java}
Using extensions to the MIPS architecture, Cibyl allows for efficient
invocation of system calls. In most operating systems, system calls are
invoked through a register-indexed table and uses fixed registers for arguments.
This approach is suboptimal for two reasons.  First, the compiler cannot
freely schedule registers around the system calls. Second, the invocation is
done through a lookup-table, which takes space in the executable and is slower
than calling statically known addresses. This effect is aggravated in Java
since indirect function calls are not allowed.

We have implemented an efficient scheme to allow native Java functionality to
be invoked with close to zero overhead.  We achieve this by using special
instruction encodings for passing system call arguments and invoking Java
methods, which is possible since we are not bound by the restrictions imposed
by the pure MIPS instruction set. Technically, the implementation uses the
ability of GCC inline assembly to emit register numbers into the generated
code.

Figure~\ref{fig:cibyl:syscalls} show the extended MIPS instructions generated
for a sequence of instructions which use native Java functionality.  Only the
return value is fixed to a register (\texttt{v0}), otherwise the compiler is
able to schedule registers freely. The \texttt{get\_image} method call uses a
constant argument, which is assigned to a temporary register. Since the
registers can be chosen freely, the return value of the first method call (in
\texttt{v0}) is directly used as an argument to the second call
(\texttt{new\_sprite}). The system call invocations are translated into calls
of static Java methods in the generated system call wrappers.

\begin{figure*}[htb]
  \begin{center}
    \footnotesize
\begin{verbatim}
la                 t0, string_address     # image = get_image("/test.png")
syscall_argument   t0
syscall_invoke     35 (get_image)
syscall_argument   v0                     # p->sprite = new_sprite(image);
syscall_invoke     20 (new_sprite)
sw                 v0,12(a0)
\end{verbatim}
  \end{center}
  \caption{System call handling in Cibyl}
  \label{fig:cibyl:syscalls}
\end{figure*}

\subsection{Runtime Support}
To support integration with native Java, Cibyl allows passing Java objects to
and from C code via integer handles. The runtime environment keeps a registry
with mappings between Java objects and handles. Objects are always accessed
through the registry.

The C API to access native Java classes is semi-generated, with only the C
function prototype being added manually. The C API is structured so that the
Java class name and method name can be extracted from the name of the
prototype and the Cibyl tools generate accesses to Java objects through the
object registry. There are a few cases where the automatic generation of
system call wrappers doesn't work, e.g., when passing Java arrays or Java
implementations of ANSI C functionality. For these, the Cibyl tools also
support inserting manual implementations of the system call wrappers.

We also implemented a subset of the ANSI C environment with file operations,
heap management, most string operations and floating point functions for
trigonometry etc. Most of this is implemented in plain C, with helper
functions in Java. This is provided as a \texttt{libc.a} library file, but
since unused functions are pruned it does not add more to the binary size than
needed.

\section{Evaluation}
\label{sec:cibyl:evaluation}
The Cibyl tools are written in Python, with support libraries for the compiled
programs written in C, and the runtime environment in Java. Since we have
utilized standard tools whenever possible (e.g., to read and parse ELF files
and compile Java assembly to bytecode), the Cibyl tools themselves are fairly
small. The tools totally comprise around 3200 lines of code including
comments, of which 2600 lines implements the binary translator and the rest
implements the generation of system call wrappers and C headers for the system
calls.

The runtime support consists of 357 lines of Java code (including comments)
and less than 100 lines of C and assembly code (which sets up the environment
and calls global constructors). Most of the runtime code implements support
for byte and short-sized memory access and the object registry. In addition,
the ANSI C environment is currently 1297 lines of C code 368 lines of Java and
the soft-float implementation consists of 357 lines of C code and 372 lines of
Java.

We have so far ported a number of applications to Cibyl, including several
games. For some of these, we have ported the applications to the J2ME C API,
and the porting effort then varies depending on how well the API maps to the
J2ME API. For others, we have instead left the applications completely
untouched and instead implemented the API in Java as a system call set or in
C, using the J2ME API. In most cases, the porting process has been
straightforward, mostly consisting of adapting the build system to Cibyl and
reimplementing the API-dependent parts for graphics, sound and keyboard input.

The largest Cibyl application we know of is RoadMap~\cite{shabtai07roadmap}, a
GPS navigation software which was previously available for UNIX and PocketPC
platforms. RoadMap uses the Cibyl syscall facilities quite extensively, since
the bluetooth GPS support uses an external Java library, and also employs a
lot of floating point operations. The RoadMap implementation consists of
around 40000 lines of C code and 1300 lines of Java and was completely
implemented by an external developer.

\subsection{Benchmarks}
To see the performance and code size impact compared to native Java we have
implemented a number of benchmarks both in Java and in C. The benchmarks are
implementations of the Game of Life and the A* algorithms. Both benchmarks are
implemented in a Cibyl-friendly way, i.e., using 32-bit values for data
accessed in the critical path. We measure the time it takes to run the actual
algorithm, disregarding startup time. The benchmarks were executed on a 600MHz
Intel Pentium M processor running Debian GNU/Linux. The time is the average of
10 runs.

\begin{table*}[thb]
  \centering
  \caption[Class size for Cibyl and native Java]{The size of compiled classes for Cibyl and native Java, in bytes. The
    MIPS size is the size of the code segment only. Cibyl
    size is split in three categories: the program itself, system call wrappers and the C runtime.}
  \small
  \begin{tabular}{lrr|r|r}
  & & Cibyl bytes \\
  Benchmark  &  Lines of code & (program/syscalls/  & Java bytes & MIPS bytes \\
  & & C runtime)\\
  \hline
  Life       & 115            &  7882 / 3806 / 5394           & 1853       & 5204 \\
  A*         & 879            & 13205 / 3619 / 5394           & 21897      & 6836 \\
  \hline
  ADPCM      & 788            & 11029 / 4486 / 5394           &            & 5572 \\
  PEGWIT     & 7175           & 83858 / 5313 / 5394           &            & 54004 \\
\end{tabular}
\label{tab:size}
\end{table*}

We also ran two of the benchmarks from the Mediabench benchmark
suite~\cite{lee97mediabench}, ADPCM and PEGWIT. ADPCM is a speech compression
benchmark and PEGWIT performs public key encryption and authentication. We
limited the selection to these two since many of the benchmarks in the suite
uses floating point operations, which is currently stabilizing in Cibyl. These
benchmarks compare the native C implementation to the recompiled Cibyl version
and we measure only the execution of the actual algorithm (startup time is
disregarded).

The Game of Life benchmark primarily stresses the memory system with loops of
matrix updates. The Java implementation is a straight port from the C
implementation, using static methods and static variables for global C
variables. We ran the benchmark for 1000 iterations on a 100x100 field.  The
logic and structure for both A* implementations is the same, but the Java
implementation uses multiple classes in the way a native implementation would.
The graph search visits 8494 nodes. The A* benchmark stresses function calls,
memory allocation and dereferencing of pointers and references.

\begin{table*}[htb]
  \centering
  \caption[A* and game of life benchmarks]{Performance results for the A* and game of life benchmarks.}
  \begin{tabular}{lrr|r}
                & Cibyl     & Native Java &            \\
    JVM (Life)  & (seconds) & (seconds)   & Slowdown   \\
    \hline
    Gij         & 25.6131   & 28.5493     & 0.90       \\
    SableVM     & 20.0237   & 18.9271     & 1.06       \\
    Kaffe       &  2.3426   &  2.3020     & 1.02       \\
    Sun JDK     &  1.1712   &  1.3431     & 0.87       \\
    \hline
    \hline
         & Cibyl     & Native Java &        \\
JVM (A*) & (seconds) & (seconds) & Slowdown \\
\hline
Gij      & 1.2268    & 0.6238    & 1.97     \\
SableVM  & 0.9560    & 0.5320    & 1.80     \\
Kaffe    & 0.1089    & 1.0649    & 0.10     \\
Sun JDK  & 0.1390    & 0.1002    & 1.39
  \end{tabular}
  \label{tab:performance}
\end{table*}

We have executed the benchmarks with the Sun JDK 1.5.0~\cite{sunjdk}, the
Kaffe JVM~\cite{kaffe}, the SableVM~\cite{gagon03sablevm} interpreter and the
GNU Java bytecode interpreter (gij)~\cite{gcj}. Gij and SableVM are bytecode
interpreters, and therefore similar to the K Virtual Machine~\cite{sun00kvm}
common in low-end and older J2ME devices.  Kaffe uses a fairly simple
Just-in-Time compiler, and is similar to the more recent CLDC HotSpot virtual
machine~\cite{sun05cldchotspot}. The Sun JDK has the most advanced virtual
machine, which will not be available in J2ME devices in the near future. The C
code was compiled with GCC 4.1.2 and optimizes for size with the \texttt{-Os}
switch. Cibyl optimization was turned on, and the Cibyl peephole optimizer was
used to post-process the Java bytecode assembly file.

\subsection{Code Size}
Table~\ref{tab:size} shows the size of the benchmarks for both Cibyl and
native Java. The size of the compiled Cibyl programs can be split in three
parts: the size of the recompiled program itself, the size of the generated
Java wrappers including support classes and the size of the runtime
environment. The runtime environment size is constant for all programs,
whereas the size of the system call wrappers will depend on the number of
system calls referenced by the program. The C library support consumes a
significant part of the code for smaller programs, e.g., the \texttt{printf}
implementation alone consumes 2.5KB in the compiled Cibyl class.

For the A* and game of life benchmarks, we can see that the size of the actual
program is within a factor of two of the Java implementation (and for the A*
benchmark lower than the Java implementation). Compared to the MIPS code, the
size overhead is between 2 and 4 (including the runtime support) for all the
benchmarks. For larger programs, we expect the size overhead compared to Java
will be small.

\subsection{Performance}
Table~\ref{tab:performance} shows the performance results for the A* and game
of life benchmarks.  The first thing that can be noted is the large
performance difference between the JVMs, with a 10-30 times performance
difference within the same language in the most extreme cases. Secondly, we
can see that the performance of Cibyl is within a factor of 2 of the native
Java implementation, and in one case clearly outperforms Java.

For the game of life benchmark, GCC was able to optimize the main part of the
code very well and this leads to less overhead for Cibyl, which is within 27\%
of the native Java implementation. A number of interesting properties are
shown in the A* benchmark. For all JVMs except Kaffe, this benchmark shows
worse results for Cibyl. As with game of life, the two JIT compilers fares
better on Cibyl than the pure interpreters.  Interesting to note is the
extremely good results for Kaffe, which is the fastest result on Cibyl and
almost 10 times faster than Java on the same virtual machine, much because of
the bad results of the Java implementation.

We believe this is caused by the differences in the generated bytecode. The
Java version uses invocations of virtual methods and accesses object fields,
whereas Cibyl uses static methods similar to C code. The Kaffe JVM clearly has
difficulties with invoking virtual methods and interfaces in the Java
implementation, while it optimizes well for the simpler bytecode (static
methods) which Cibyl generates.

\begin{table*}[htb]
  \centering
  \caption[Mediabench benchmark results]{Performance results in seconds for the mediabench benchmarks. The Cibyl
    results were obtained with the Sun JVM.}
  \small
  \begin{tabular}{lrrr|rr}
              & Cibyl     & Cibyl no file ops & Native \\
    Benchmark & (seconds) & (seconds)         & (seconds)  & Slowdown & Slowdown no \\
              &           &                   &            &          & file ops. \\
    \hline
    ADPCM     & 0.821     & N/A               & 0.031      & 26.0     & N/A \\
    PEGWIT    & 1.307     & 0.288             & 0.051      & 25.62    & 5.64\\
  \end{tabular}
  \label{tab:mediabench_performance}
\end{table*}

When comparing recompiled Cibyl code with native C code, there is a large
slowdown as shown in Table~\ref{tab:mediabench_performance}. However, this is
mostly because of the current inefficient implementation file operations. For
ADPCM, input read with \texttt{fread} is the culprit and for PEGWIT, producing
the output with \texttt{fwrite} causes most of the degradation. By making
\texttt{fwrite} a no-op, PEGWIT finishes in less than 0.3 seconds, which
suggests that improving the performance of file operations should be a future
priority.

\section{Related Work}
\label{sec:cibyl:related_work}
NestedVM~\cite{alliet04nestedvm} also performs a binary translation of MIPS
binaries to Java bytecode and therefore has many similarities with Cibyl.
However, NestedVM has different goals than Cibyl. The main focus of NestedVM
is to recompile and run insecure native binaries in a secure VM. In contrast,
Cibyl offers an alternative environment on Java-based platforms. NestedVM has
a UNIX-compatibility layer to support recompilation and execution of existing
UNIX tools, and consequently requires a larger runtime environment.

Technically, NestedVM and Cibyl are also different. To support sparse memory,
NestedVM uses a matrix memory representation whereas Cibyl uses a vector. For
the embedded applications Cibyl target, the improved performance of the vector
representation is more important than the ability to support large memories
effectively. NestedVM also uses class-variables as register representation
whereas Cibyl uses local variables, which gives more efficient and compact
bytecode. The use of architectural extensions also separates Cibyl from
NestedVM, and Cibyl uses a hybrid software floating point implementation while
NestedVM implements the MIPS FPU instructions.

There are also a few compilers which generate Java bytecode directly.
Axiomatic solutions~\cite{axiomsol} has a compiler for a subset of C which
generates Java bytecode, and the University of Queensland Binary Translator
project has developed a Java bytecode backend for GCC~\cite{cifuentes00UAB}.
Compared to the Axiomatic solutions compiler, Cibyl provides full support for
the C language can leverage the GCC optimization framework. The Java bytecode
backend is not part of GCC and therefore requires a significant effort to
track mainline development. In contrast, maintenance of Cibyl is independent
of GCC and benefits automatically from GCC updates and Cibyl also provides a
complete development environment for J2ME devices.

% Binary translation, dynamic/static, Fx!32 etc
The area of binary translation can roughly be separated into two areas: static
and dynamic binary translators, with Cibyl being a static binary translator. A
cited problem with static translators is separating code from
data~\cite{altman00welcome} but being a development environment, Cibyl does
not have these problems. By retaining relocation and symbol information from
the compiler and using static linking, Cibyl cleanly separates code from data
and can prune unused functions and executing data or rewriting code is not
possible in Java. Dynamic binary translators, performing the translation
during runtime, avoid problems with static binary translation, but typically
targets other problems than Cibyl such as translating unmodified binaries from
one architecture to another~\cite{hookway97fx32}, performing otherwise
difficult optimizations~\cite{bala00dynamo} or implementing debug and
inspection support~\cite{valgrind}. Since Cibyl targets memory-constrained
embedded systems, the large runtime support system needed for a dynamic
translator would be a disadvantage.

% Dynamic Binary Translation and Optimization

\section{Conclusions}
\label{sec:cibyl:conclusions}
In this paper, we have presented the Cibyl binary translation environment. We
have described how extensions to the MIPS architecture and use of the ABI can
help bring the performance close to that of native Java implementations. We
have further described how Java functionality can be integrated into C
programs in the Cibyl environment efficiently and with small effort. The
approach taken by Cibyl should be comparable in performance to a dedicated
compiler backend for Java bytecode, with much less effort.  We believe that
Cibyl can fill an important niche for porting of existing programs in C and
other languages to the J2ME platform, but also for development of new projects
where Java might not be the ideal language. We also expect that the small size
of the Cibyl tools make it easy to maintain in the long run. Since we use
unmodified standard tools, Cibyl benefits from improvements and new tool
versions without the need to continuously track development.

There are multiple possible directions for future research on Cibyl. First, to
reduce the overhead of function calls, functions which are called in a chain
(determined through profiling) can be colocated to one Java method with a
common entry point. Second, to further improve performance and reduce size, we
are planning an implementation of register value tracking. Third, to better
support debugging, we are investigating an implementation of GDB support for
Cibyl. Fourth, runtime libraries for C++ and other languages would further
increase the applicability of Cibyl.


\section*{Acknowledgements and Availability}
This work was partly funded by The Knowledge Foundation in Sweden under a
research grant for the project ``Blekinge - Engineering Software Qualities
(BESQ)'' (\url{http://www.bth.se/~besq}). Cibyl is free software under the GNU
GPL and can be downloaded from \url{http://cibyl.googlecode.com}.

%%% Local Variables:
%%% mode: latex
%%% TeX-master: t
%%% End:

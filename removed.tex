
Figure~\ref{fig:introduction:cibyl-memory-access} illustrates two alternative
memory organization. Note that the byte-sized memory representation requires
two vector accesses while the integer-sized representation only needs one. In
the byte-vector case, byte-sized accesses can be done directly with only one
memory access while integer-sized accesses (32-bits) can be done with a single
vector load in the other case. Integer-sized accesses are also proportionally
more expensive with the byte memory vector than byte-sized accesses with the
integer memory vector since the former requires three bitwise \textbf{shift}'s
and four bitwise \textbf{or}'s while the latter only needs one \textbf{shift}
and one \textbf{and}.

\begin{figure}[t]
  \begin{center}
    \includegraphics[width=0.90\linewidth]{figures/introduction/cibyl-memory-access}
  \end{center}
  \caption[Translating memory accesses]{Two alternatives for translating memory accesses, illustrated with
    reading a 16-bit short on address 2. The left part of the figure shows an
    access with a byte-vector memory representation and the right the same
    accesss with an integer-vector memory representation.}
  \label{fig:introduction:cibyl-memory-access}
\end{figure}

Another general problem is unsigned arithmetic. All integer arithmetic
instructions in Java bytecode operates on signed values whereas MIPS assembly
(and most other instruction sets) has instructions for both signed and
unsigned arithmetic. For most instructions this is not a problem, since e.g.,
signed addition of two unsigned values will yield the same result as unsigned
addition. The main sources of problems are unsigned multiplications, divisions and
comparisons. The easiest way to solve these problems is to extend the 32-bit
value into a 64-bit value, mask away the top 32 bits, perform the operation
and crop the result to 32-bits again.

Finally, while most MIPS instructions can be translated with a 1-1 to Java
bytecode there are some exceptions. These instructions are multiplications,
divisions, non-branch comparison instructions and register-indirect branches
and calls. MIPS multiplications generate a 64-bit result in two registers
while the corresponding Java bytecode instruction generate a 32-bit result.
Similarly, the MIPS \texttt{div}/\texttt{divu} instructions generate both the
result and the remainder in two registers, while these are two instructions in
Java bytecode. The MIPS \texttt{slt} and \texttt{sltu} instructions assign a
register to the result of a comparison, which has no corresponding Java
bytecode instruction. These instructions can be translated to a set of
multiple Java bytecode instructions.

Register-indirect branches and calls are more difficult to implement in Java.
Because of the security requirements, Java bytecode does not have computed
gotos. The closest equivalent is two \texttt{switch} instructions, which can
be used to switch on a passed address. Both NestedVM and Cibyl use this
technique although in a slightly different way, illustrated in
Figure~\ref{fig:introduction:cibyl-nestedvm-jalr}. Cibyl use a 1-1 mapping
between C functions (ELF entry points) and Java methods. To support indirect
calls, Cibyl generates a special method that contains a single big switch with
calls to all possible targets of indirect calls. As shown in the left part of
the figure, an indirect call from \texttt{fn1} to \texttt{fn2} will be a
three-step process through the indirect call table.

\begin{figure}[t]
  \begin{center}
    \includegraphics[width=0.90\linewidth]{figures/introduction/cibyl-nestedvm-jalr}
  \end{center}
  \caption{Indirect call handling}
  \label{fig:introduction:cibyl-nestedvm-jalr}
\end{figure}

NestedVM does not use a 1-1 mapping between C functions and Java methods.  The
method split, which is needed because of a Java 64KB method size limit, can
instead be done at any location (with heuristics to avoid splitting functions
unnecessary). To handle method calls in general, NestedVM therefore uses a flat
structure with a top-level call table not tied to the C function calls as
illustrated in the right part of the figure. In the call, NestedVM will return
from \texttt{fn1} to the global call table and from there continue to
\texttt{fn2}.

There are advantages with both approaches. The 1-1 mapping used in Cibyl is
closer to the original program, providing meaningful backtraces on crashes. It
also allows better performance for the common case of direct calls (where
NestedVM sometimes still needs to use the indirect method). The NestedVM
approach instead makes it easier to support very big C functions (over 64KB of
compiled code), \texttt{longjmp} (jumping to a saved stack context, possibly
skipping called functions) and badly written assembly code.

%%% Local Variables: 
%%% mode: latex
%%% TeX-master: t
%%% End: 

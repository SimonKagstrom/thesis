% Intro

Computer hardware and software evolve very fast. With the advent of
chip-multi\-processors and symmetric multithreading, multiprocessor hardware
configurations are becoming prevalent. For software, new hardware and
requirements such as security, performance and maintainability drive the
development of new runtime environments, virtual machines and programming
methodologies. These trends present problems when porting legacy
software. Multiprocessor hardware require ports of uniprocessor operating
system kernels while new software environments might require that programs
have to be ported to different languages.

This thesis examines the tradeoff between performance and development effort
for software porting with case studies in operating system porting to
multiprocessors and tool support for porting C and C++ applications to Java
virtual machines. The thesis consists of seven papers. The first paper is a
survey of existing multiprocessor development approaches and focuses on the
tradeoff between performance and implementation effort. The second and third
papers describe the evolution a traditional lock-based multiprocessor port,
going from a serialized ``giant locked'' port and evolving into a
coarse-grained implementation. The fourth paper instead presents an
alternative porting approach which aims to minimize development effort. The
fifth paper describes a tool for efficient instrumentation of programs, which
can be used during the development of large software systems such as operating
system kernels. The sixth and seventh papers finally describe a binary
translator which translates MIPS binaries into Java bytecode to allow
low-effort porting of C and C++ applications to Java virtual machines.


% Contributions

The first main contributions of this thesis is an in-depth investigation of
the techniques used when porting operating system kernels to multiprocessors,
focusing on development effort and performance. The traditional approach used
in the second and third papers required longer development time than expected,
and the alternative approach in the fourth paper can therefore be preferable
in some cases. The second main contribution is the development of a binary
translator that targets portability of C and C++ applications to J2ME
devices. The last two papers show that the approach is functional and has good
enough performance to be feasible in real-life situations.

% Conclusions

